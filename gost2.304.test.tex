\documentclass{article}

% !TeX spellcheck = ru_RU
% !TeX program = xelatex

\usepackage[mainfontsize=3.5mm]{gost2.304}
\usepackage{interfaces}
\usepackage{tikz}
\usepackage{interfaces-tikz}
\usepackage[a4 paper, portrait, tmargin=10mm, bmargin=10mm, inner=20mm, outer=20mm, headheight=0mm, headsep=0mm, footskip=0mm]{geometry}
\usepackage{unicode-math}

\begin{document}

  \parindent=0pt\relax
  \papergraduate

  %\eskdfont
  \tiny Tiny 999 \bfseries Tiny 999 - должны быть одинаково утолщёнными (1.8 мм - допустим только для шрифта Б)
  \fontsize{1mm}{1mm}\selectfont И это - не 1 мм, а 1.8 мм.

  \upshape\mdseries\normalsize Нормальный	
  \small small
  \footnotesize Footnotesize
  \scriptsize Scriptsize
  \tiny Tiny
  \large large
  \Large Large
  \LARGE LARGE
  \huge huge
  \Huge Huge

  \fontsize{40mm}{68mm}\selectfont 40 Ш
  \textbf{40}
  {\slshape 40}
  {\slshape\bfseries 40}

  \NewDocumentCommand \DrawChars {m} {
    \begin{tikzpicture}
    \papergraduate (-10mm,0mm) (160mm,20mm);
    \node at (0mm,0mm) [anchor=south west,inner sep=0pt,outer sep=0pt]
    {
      #1
    };
    \end{tikzpicture}
  }

  \newpage
  \eskdfont \eskdfontsize{14mm} \selectfont\mdseries
  Раздел 3 ГОСТ 2.304-81. Шрифт типа А
  
  {
    \slshape
    \DrawChars{АБВГДЕЖЗИЙКЛМНОПР}
    \DrawChars{СТУФХЦЧШЩЪЫЬЭЮЯ}
    \DrawChars{абвгдежзийклмнопрст}
    \DrawChars{уфхцчшщъыьэюя}
  }

  {
    \DrawChars{АБВГДЕЖЗИЙКЛМНОПР}
    \DrawChars{СТУФХЦЧШЩЪЫЬЭЮЯ}
    \DrawChars{абвгдежзийклмнопрст}
    \DrawChars{уфхцчшщъыьэюя}
  }

  \newpage
  \eskdfont \eskdfontsize{10mm}\selectfont\bfseries
  Шрифт типа Б

  {
    \slshape
    \DrawChars{АБВГДЕЖЗИЙКЛМНОПР}
    \DrawChars{СТУФХЦЧШЩЪЫЬЭЮЯ}
    \DrawChars{абвгдежзийклмнопрст}
    \DrawChars{уфхцчшщъыьэюя}
  }

  {
    \DrawChars{АБВГДЕЖЗИЙКЛМНОПР}
    \DrawChars{СТУФХЦЧШЩЪЫЬЭЮЯ}
    \DrawChars{абвгдежзийклмнопрст}
    \DrawChars{уфхцчшщъыьэюя}
  }

  \newpage
  \eskdfont \eskdfontsize{14mm} \selectfont\mdseries
  Раздел 4 ГОСТ 2.304-81. Шрифт типа А

  {
    \slshape
    \DrawChars{ABCDEFGHIJKLMN}
    \DrawChars{OPQRSTUVWXYZ}
    \DrawChars{abcdefghijklmnop}
    \DrawChars{qrstuvwxyz}
  }

  {
    \DrawChars{ABCDEFGHIJKLMNO}
    \DrawChars{PQRSTUVWXYZ}
    \DrawChars{abcdefghijklmnopq}
    \DrawChars{rstuvwxyz}
  }

  \newpage
  \eskdfont \eskdfontsize{10mm}\selectfont\bfseries
  Шрифт типа Б

  {
    \slshape
    \DrawChars{ABCDEFGHIJKLMN}
    \DrawChars{OPQRSTUVWXYZ}
    \DrawChars{abcdefghijklmnop}
    \DrawChars{qrstuvwxyz}
  }

  {
    \DrawChars{ABCDEFGHIJKLMN}
    \DrawChars{OPQRSTUVWXYZ}
    \DrawChars{abcdefghijklmnopq}
    \DrawChars{rstuvwxyz}
  }

  \newpage
  \eskdfont \eskdfontsize{14mm} \selectfont\mdseries
  Раздел 5 ГОСТ 2.304-81. Шрифт типа А

  {
    \slshape
    \DrawChars{\Alpha \Beta \Gamma \Delta \Epsilon \Zeta \Eta \Theta \Iota \Kappa \Lambda \Mu \Nu \Xi}
    \DrawChars{\Omicron \Pi \Rho \Sigma \Tau \Upsilon \Phi \Chi \Psi \Omega}
    \DrawChars{\alpha \beta \gamma \delta \epsilon \zeta \eta \theta \iota \kappa \lambda \mu \nu}
    \DrawChars{\xi \omicron \pi \rho \sigma \tau \upsilon \phi \chi \psi \omega}
  }

  {
    \DrawChars{\Alpha \Beta \Gamma \Delta \Epsilon \Zeta \Eta \Theta \Iota \Kappa \Lambda \Mu \Nu \Xi \Omicron}
    \DrawChars{\Pi \Rho \Sigma \Tau \Upsilon \Phi \Chi \Psi \Omega}
    \DrawChars{\alpha \beta \gamma \delta \epsilon \zeta \eta \theta \iota \kappa \lambda \mu \nu \xi \omicron}
    \DrawChars{\pi \rho \sigma \tau \upsilon \phi \chi \psi \omega}
  }

  \newpage
  \eskdfont \eskdfontsize{10mm}\selectfont\bfseries
  Шрифт типа Б

  {
    \slshape
    \DrawChars{\Alpha \Beta \Gamma \Delta \Epsilon \Zeta \Eta \Theta \Iota \Kappa \Lambda \Mu \Nu}
    \DrawChars{\Xi \Omicron \Pi \Rho \Sigma \Tau \Upsilon \Phi \Chi \Psi \Omega}
    \DrawChars{\alpha \beta \gamma \delta \epsilon \zeta \eta \theta \iota \kappa \lambda \mu}
    \DrawChars{\nu \xi \omicron \pi \rho \sigma \tau \upsilon \phi \chi \psi \omega}
  }

  {
    \DrawChars{\Alpha \Beta \Gamma \Delta \Epsilon \Zeta \Eta \Theta \Iota \Kappa \Lambda \Mu \Nu}
    \DrawChars{\Xi \Omicron \Pi \Rho \Sigma \Tau \Upsilon \Phi \Chi \Psi \Omega}
    \DrawChars{\alpha \beta \gamma \delta \epsilon \zeta \eta \theta \iota \kappa \lambda \mu \nu}
    \DrawChars{\xi \omicron \pi \rho \sigma \tau \upsilon \phi \chi \psi \omega}
  }

  \newpage
  \eskdfont \eskdfontsize{14mm} \selectfont\mdseries
  Раздел 6 ГОСТ 2.304-81. Шрифт типа А

  {
    \DrawChars{1234567890}
  }
  {
    \slshape
    \DrawChars{1234567890}
  }
  \newcounter{digit}%
  {
    \DrawChars{%
      \setcounter{digit}{1}\Roman{digit}
      \setcounter{digit}{3}\Roman{digit}
      \setcounter{digit}{4}\Roman{digit}
      \setcounter{digit}{6}\Roman{digit}
      \setcounter{digit}{8}\Roman{digit}
      \setcounter{digit}{9}\Roman{digit}
    }
  }
  {
    \slshape
    \DrawChars{%
      \setcounter{digit}{1}\Roman{digit}
      \setcounter{digit}{3}\Roman{digit}
      \setcounter{digit}{4}\Roman{digit}
      \setcounter{digit}{6}\Roman{digit}
      \setcounter{digit}{8}\Roman{digit}
      \setcounter{digit}{9}\Roman{digit}
    }
  }

  \eskdfont \eskdfontsize{10mm}\selectfont\bfseries
  Шрифт типа Б

  {
    \DrawChars{1234567890}
  }
  {
    \slshape
    \DrawChars{1234567890}
  }
  {
    \DrawChars{%
      \setcounter{digit}{1}\Roman{digit}
      \setcounter{digit}{3}\Roman{digit}
      \setcounter{digit}{4}\Roman{digit}
      \setcounter{digit}{6}\Roman{digit}
      \setcounter{digit}{8}\Roman{digit}
      \setcounter{digit}{9}\Roman{digit}
    }
  }
  {
    \slshape
    \DrawChars{%
      \setcounter{digit}{1}\Roman{digit}
      \setcounter{digit}{3}\Roman{digit}
      \setcounter{digit}{4}\Roman{digit}
      \setcounter{digit}{6}\Roman{digit}
      \setcounter{digit}{8}\Roman{digit}
      \setcounter{digit}{9}\Roman{digit}
    }
  }

\end{document}