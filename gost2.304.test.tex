\documentclass[trace=off]{article}

% !TeX spellcheck = ru_RU
% !TeX program = xelatex

\usepackage[mainfontsize=3.5mm]{gost2.304}
%\usepackage{mathspec}
\usepackage{interfaces}
\usepackage{tikz}
\usepackage{interfaces-tikz}
\usepackage[a4 paper, portrait, tmargin=10mm, bmargin=10mm, inner=20mm, outer=20mm, headheight=0mm, headsep=0mm, footskip=0mm]{geometry}
\usepackage{unicode-math}

\setmathfont
  [
    range = {\mathit,\mathup,\mathbin,\mathcal,\mathbfcal,\mathscr},
    math-style = TeX,
    %Path = fonts/,
    %Extension = .ttf,
    %Font = Mipgost,
    FakeBold = 1.4
  ]
  {Mipgost}
%\setmathfont[range={\mathit/{greek,Greek,latin,Latin,num,misc},\mathup,\mathbin,\mathcal,\mathbfcal,\mathscr}]{Mipgost}
%\setmathfont{eskdfont}
%\setmathfont[range={\mathit}]{eskdfont}
%\setmathfont[range={\mathup}]{eskdfont}
%\setmathfont[range={\mathbin}]{eskdfont}

\begin{document}

  \parindent=0pt\relax
  \raggedright
  \papergraduate

  \makeatletter
  \f@family\,\f@series\,\@roman{4}\, \eskdfont \f@family\,\@roman{4}\,\@roman{6}\,\@roman{12}

  \eskdfontsize{1.8mm}\selectfont
  Tiny 999 \bfseries Tiny 999 - должны быть одинаково утолщёнными (1.8 мм - допустим только для шрифта Б \f@series)
  \eskdfontsize{1mm}\selectfont И это - не 1 мм, а 1.8 мм.

  \normalsize\upshape\mdseries Нормальный
  \small small
  \footnotesize Footnotesize
  \scriptsize Scriptsize
  \tiny Tiny
  \large large
  \Large Large
  \LARGE LARGE
  \huge huge
  \Huge Huge
  
  \eskdfontsize{5mm}\selectfont
  \newlength\ldn
  \ldn=1em 1\,em равен \the\ldn,
  \ldn=1ex 1\,ex равен \the\ldn,
  \ldn=5mm 5\,mm равно \the\ldn.
  \f@family
  
  \eskdfontsize{40mm}\selectfont 40
  {\slshape 40}
  \textbf{40}
  {\slshape\bfseries 40}

  \NewDocumentCommand \DrawChars {m} {
    \begin{tikzpicture}
    \papergraduate (-5mm,0mm) (160mm,20mm);
    \node at (0mm,0mm) [anchor=south west,inner sep=0pt,outer sep=0pt]
    {
      #1
    };
    \end{tikzpicture}
  }

  \newpage
  \eskdfont \eskdfontsize{14mm} \selectfont\mdseries
  Раздел 3 ГОСТ 2.304-81. Шрифт типа А
  
  {
    \slshape
    \DrawChars{АБВГДЕЖЗИЙКЛМНОПР}
    \DrawChars{СТУФХЦЧШЩЪЫЬЭЮЯ}
    \DrawChars{абвгдежзийклмнопрст}
    \DrawChars{уфхцчшщъыьэюя}
  }

  {
    \DrawChars{АБВГДЕЖЗИЙКЛМНОПР}
    \DrawChars{СТУФХЦЧШЩЪЫЬЭЮЯ}
    \DrawChars{абвгдежзийклмнопрст}
    \DrawChars{уфхцчшщъыьэюя}
  }

  \newpage
  \eskdfont \eskdfontsize{10mm}\selectfont\bfseries
  Шрифт типа Б

  {
    \slshape
    \DrawChars{АБВГДЕЖЗИЙКЛМНОПР}
    \DrawChars{СТУФХЦЧШЩЪЫЬЭЮЯ}
    \DrawChars{абвгдежзийклмнопрст}
    \DrawChars{уфхцчшщъыьэюя}
  }

  {
    \DrawChars{АБВГДЕЖЗИЙКЛМНОПР}
    \DrawChars{СТУФХЦЧШЩЪЫЬЭЮЯ}
    \DrawChars{абвгдежзийклмнопрст}
    \DrawChars{уфхцчшщъыьэюя}
  }

  \newpage
  \eskdfont \eskdfontsize{14mm} \selectfont\mdseries
  Раздел 4 ГОСТ 2.304-81. Шрифт типа А

  {
    \slshape
    \DrawChars{ABCDEFGHIJKLMN}
    \DrawChars{OPQRSTUVWXYZ}
    \DrawChars{abcdefghijklmnop}
    \DrawChars{qrstuvwxyz}
  }

  {
    \DrawChars{ABCDEFGHIJKLMNO}
    \DrawChars{PQRSTUVWXYZ}
    \DrawChars{abcdefghijklmnopq}
    \DrawChars{rstuvwxyz}
  }

  \newpage
  \eskdfont \eskdfontsize{10mm}\selectfont\bfseries
  Шрифт типа Б

  {
    \slshape
    \DrawChars{ABCDEFGHIJKLMN}
    \DrawChars{OPQRSTUVWXYZ}
    \DrawChars{abcdefghijklmnop}
    \DrawChars{qrstuvwxyz}
  }

  {
    \DrawChars{ABCDEFGHIJKLMN}
    \DrawChars{OPQRSTUVWXYZ}
    \DrawChars{abcdefghijklmnopq}
    \DrawChars{rstuvwxyz}
  }

  \newpage
  \eskdfont \eskdfontsize{14mm} \selectfont\mdseries
  Раздел 5 ГОСТ 2.304-81. Шрифт типа А

  {
    \slshape
    \DrawChars{\Alpha \Beta \Gamma \Delta \Epsilon \Zeta \Eta \Theta \Iota \Kappa \Lambda \Mu \Nu \Xi}
    \DrawChars{\Omicron \Pi \Rho \Sigma \Tau \Upsilon \Phi \Chi \Psi \Omega}
    \DrawChars{\alpha \beta \gamma \delta \epsilon \zeta \eta \theta \iota \kappa \lambda \mu \nu}
    \DrawChars{\xi \omicron \pi \rho \sigma \tau \upsilon \phi \chi \psi \omega}
  }

  {
    \DrawChars{\Alpha \Beta \Gamma \Delta \Epsilon \Zeta \Eta \Theta \Iota \Kappa \Lambda \Mu \Nu \Xi \Omicron}
    \DrawChars{\Pi \Rho \Sigma \Tau \Upsilon \Phi \Chi \Psi \Omega}
    \DrawChars{\alpha \beta \gamma \delta \epsilon \zeta \eta \theta \iota \kappa \lambda \mu \nu \xi \omicron}
    \DrawChars{\pi \rho \sigma \tau \upsilon \phi \chi \psi \omega}
  }

  \newpage
  \eskdfont \eskdfontsize{10mm}\selectfont\bfseries
  Шрифт типа Б

  {
    \slshape
    \DrawChars{\Alpha \Beta \Gamma \Delta \Epsilon \Zeta \Eta \Theta \Iota \Kappa \Lambda \Mu \Nu}
    \DrawChars{\Xi \Omicron \Pi \Rho \Sigma \Tau \Upsilon \Phi \Chi \Psi \Omega}
    \DrawChars{\alpha \beta \gamma \delta \epsilon \zeta \eta \theta \iota \kappa \lambda \mu}
    \DrawChars{\nu \xi \omicron \pi \rho \sigma \tau \upsilon \phi \chi \psi \omega}
  }

  {
    \DrawChars{\Alpha \Beta \Gamma \Delta \Epsilon \Zeta \Eta \Theta \Iota \Kappa \Lambda \Mu \Nu}
    \DrawChars{\Xi \Omicron \Pi \Rho \Sigma \Tau \Upsilon \Phi \Chi \Psi \Omega}
    \DrawChars{\alpha \beta \gamma \delta \epsilon \zeta \eta \theta \iota \kappa \lambda \mu \nu}
    \DrawChars{\xi \omicron \pi \rho \sigma \tau \upsilon \phi \chi \psi \omega}
  }

  \newpage
  \eskdfont \eskdfontsize{14mm} \selectfont\mdseries
  Раздел 6 ГОСТ 2.304-81. Шрифт типа А

  {
    \DrawChars{1234567890}
  }
  {
    \slshape
    \DrawChars{1234567890}
  }
  \newcounter{digit}%
  {
    \DrawChars{%
      \setcounter{digit}{1}\roman{digit}
      \setcounter{digit}{3}\Roman{digit}
      \setcounter{digit}{4}\Roman{digit}
      \setcounter{digit}{6}\Roman{digit}
      \setcounter{digit}{8}\Roman{digit}
      \setcounter{digit}{9}\Roman{digit}
    }
    \DrawChars{\@roman{4}\,\@roman{6}\,\@roman{12}\,\@roman{2014}}
  }
  {
    \slshape
    \DrawChars{%
      \setcounter{digit}{1}\Roman{digit}
      \setcounter{digit}{3}\Roman{digit}
      \setcounter{digit}{4}\Roman{digit}
      \setcounter{digit}{6}\Roman{digit}
      \setcounter{digit}{8}\Roman{digit}
      \setcounter{digit}{9}\Roman{digit}
    }
  }

  \newpage
  \eskdfont \eskdfontsize{10mm}\selectfont\bfseries
  Шрифт типа Б

  {
    \DrawChars{1234567890}
  }
  {
    \slshape
    \DrawChars{1234567890}
  }
  {
    \DrawChars{%
      \setcounter{digit}{1}\Roman{digit}
      \setcounter{digit}{3}\Roman{digit}
      \setcounter{digit}{4}\Roman{digit}
      \setcounter{digit}{6}\Roman{digit}
      \setcounter{digit}{8}\Roman{digit}
      \setcounter{digit}{9}\Roman{digit}
    }
    \DrawChars{\@roman{4}\,\@roman{6}\,\@roman{12}\,\@roman{2014}}
  }
  {
    \slshape
    \DrawChars{%
      \setcounter{digit}{1}\Roman{digit}
      \setcounter{digit}{3}\Roman{digit}
      \setcounter{digit}{4}\Roman{digit}
      \setcounter{digit}{6}\Roman{digit}
      \setcounter{digit}{8}\Roman{digit}
      \setcounter{digit}{9}\Roman{digit}
    }
  }

  \newpage
  \eskdfont \eskdfontsize{10mm} \selectfont\mdseries
  Раздел 8 ГОСТ 2.304-81. Степени и дроби

  \papergraduate
  %\setmathfont{eskdfont}
  $m = \frac{m_0}{\sqrt{1-\frac{v^2}{c^2}}}\times 2 \times 10^{23}$

  \begin{eqnarray}
    E &=& mc^2\\
    m &=& \frac{m_0}{\sqrt{1-\frac{v^2}{c^2}}}\times 2 \cdot 10^{23}
  \end{eqnarray}

\end{document}