
%\setmathfont{OpenGost Type B TT}
%\setmathfont[range={\mathit}]{OpenGost Type B TT}
%\setmathfont[range={\mathup}]{OpenGost Type B TT}
%\setmathfont[range={\mathbin}]{OpenGost Type B TT}
% меняем знаки неравенств на привычные "русские"
\let\le\leqslant
\let\ge\geqslant
% определяем привычный тангенс и котангенс
\let\tan\tg
\let\arctan\arctg
\let\cot\ctg
\let\arccot\arcctg
% заменяем знак умножения (x) на точку (по ГОСТ)
\let\times\cdot

\makeatletter

	\newpage
	
	\part{Тесты}

	\section{тестируем оформление текста по ГОСТ2.105}
	\subsection{тестируем заголовки}
	\subsubsection{уже тестируем.
		Заголовки выравниваем по левому краю, но с абзацного отступа.
		Переносы слов в заголовках не допускаются.
		Очень длинный абзац. Очень длинный абзац.
		Очень длинный абзац. Очень длинный абзац. Очень длинный абзац. Очень длинный абзац.}
	
	\paragraph{нумерованный абзац}. И его продолжение.
	Очень длинный абзац. Очень длинный абзац. Очень длинный абзац. Очень длинный абзац.
	Очень длинный абзац. Очень длинный абзац. Очень длинный абзац. Очень длинный абзац.
		
	\subparagraph{нумерация уровнем ниже}. И продолжение.
	Очень длинный абзац. Очень длинный абзац. Очень длинный абзац. Очень длинный абзац.
	Очень длинный абзац. Очень длинный абзац. Очень длинный абзац. Очень длинный абзац.
		
	Просто абзац без номера. Очень длинный абзац. Очень длинный абзац. Очень длинный абзац.
	Очень длинный абзац. Очень длинный абзац. Очень длинный абзац. Очень длинный абзац.
	Очень длинный абзац. Очень длинный абзац. Очень длинный абзац. Очень длинный абзац.

	\section{тестируем обозначения погрешностей}
	\ПГ(\frac{L}{2^{64}+\int\limits_{0}^{2\pi}f\left(x\right)dx}), \ПГ(L), \СКО(L), \ПГО(L).
	
	\ФункцияПреобразования
	\ЗначениеМеры
	\НоминальнаяФункцияПреобразования
	\НоминальноеЗначениеМеры
	\П
	\Псист
	\Псл
	\ПГ
	\ПГсист
	\М[x]
	\СрКО
	\СКО[U]
	\МПГсист(x)
	
	\СрКОсист
	\СрКОсл
	\СКОсл(I)
	
	\НАКФ
	\НАКФПГсл
	\S
	\SПГсл
	\МП
	\СрКОП
	\Пслгист
	
	\СКО[\П(U)]
	
	\ВР(I)
	\ПО
	\ПГО(x)
	\НСТБ(f)
	
	\section{тестируем оформление чисел в нормальной форме}
	$\numprint{+-3,1415927e-3.1}$~--- пробуем numprint. \numprint{-22978}, \numprint{-3.72e26},
	$\numprint[\celsius]{25}$, $\numprint[\degree]{25}$.
	
	Тестируем siunitx: \SIrange{1}{10}{\degreeCelsius}. % [mode=text, detect-all]

	\section{тестируем пакет objects}
	\object{Формуляр}{Изделие}{Наименование}
	\Формуляр\Изделие\Наименование

	\def\Разработал{\Формуляр\Разработал[1]}
	Разработал - \Разработал\ДолжностноеЛицо\Фамилия

	\object{Формуляр}{Разработал}[1]{ДолжностноеЛицо}{Фамилия}
	\Формуляр\Разработал[2]\ДолжностноеЛицо\Фамилия
	%\Формуляр\Разработал[3]\ДолжностноеЛицо\Фамилия
	%\Формуляр\Разработал[4]\ДолжностноеЛицо\Фамилия
		
	\Формуляр\Разработал\count
	--- \object{Формуляр}{Разработал}\count
	--- \@nameuse{@Формуляр@Разработал@count}

	%\Формуляр\Разработал\foreach{\Разработал}{
	%	--\Разработал\ДолжностноеЛицо\Фамилия--
	%}

	\Формуляр\Изделие\Характеристики\foreach{\Характеристика}{%
		--\Характеристика\Наименование, \Характеристика\Значение\ЕдиницаИзмерения --
		\Характеристика\Значение\Значение
	}
	
	\Формуляр\Разработал\for{\Разработал}[2][]{
		--\Разработал\ДолжностноеЛицо\Фамилия--
	}
		
	\section{Прочие тесты}
	\LaTeX{} "--- это своего рода {\fontshape{it}\selectfont{}препроцессор} текста для \TeX{} "---
	программы компьютерной вёрстки. \LaTeX{} является программируемым и
	расширяемым, что позволяет автоматизировать большую часть аспектов
	набора, включая нумерацию, перекрёстные ссылки, таблицы и изображения
	(их размещение и подписи к ним), общий вид страницы, библиографию и
	многое-многое другое. \LaTeX{} был первоначально написан Лэсли Лампортом
	в 1984-м году и стал наиболее популярным способом использования \TeX{}а;
	очень мало людей сегодня пишут на оригинальном \TeX{}е. Текущей
	версией является \LaTeXe.
	
	Проверим блоки.
	
	{\par\vspace{\parsep}\vspace{\parskip}\hbox{\vrule\hspace{0.5em}\parbox{\textwidth-0.5em}{
		Иногда используется следующий способ выделения текста:
		абзац набирается с некоторым отступом от левого поля,
		а слева от него, вровень с левым полем, печатается
		вертикальная линейка.
	}}\vspace{\parsep}\par}
	
	И ссылки: формула 2 - см. \ref{formula2}.
	
	\begin{adjustwidth}{10mm}{10mm}
		Поставщик
		\parbox[t]{15em}{\centering{}ООО <<Фирма>>\\*\hrule\smallskip\tiny{}наименование},
		именуемое в дальнейшем <<Поставщик>>, и другая фирма и другая фирма и другая фирма.
	\end{adjustwidth}
	
	$$
		2\times\Delta\tan\alpha \le 2*\Delta\tg\mu
	$$
	  
	\begin{eqnarray}
		E &=& mc^2\\
		m &=& \frac{m_0}{\sqrt{1-\frac{v^2}{c^2}}}\times 2 \times 10^{23} \label{formula2}
	\end{eqnarray}
	
	\LaTeX{} "--- это своего рода препроцессор текста для \TeX{} "---
	программы компьютерной вёрстки. \LaTeX{} является программируемым и
	расширяемым, что позволяет автоматизировать большую часть аспектов
	набора, включая нумерацию, перекрёстные ссылки, таблицы и изображения
	(их размещение и подписи к ним), общий вид страницы, библиографию и
	многое-многое другое. \LaTeX{} был первоначально написан Лэсли Лампортом
	в 1984-м году и стал наиболее популярным способом использования \TeX{}а;
	очень мало людей сегодня пишут на оригинальном \TeX{}е. Текущей
	версией является \LaTeXe.
	
	Проверим блоки.
	
	{\par\vspace{\partopsep}\hbox{\vrule\hspace{0.5em}\parbox{\textwidth-0.5em}{
		Иногда используется следующий способ выделения текста:
		абзац набирается с некоторым отступом от левого поля,
		а слева от него, вровень с левым полем, печатается
		вертикальная линейка.
	}}\par\vspace{\partopsep}}
	
	И ссылки: формула 2 - см. \ref{formula2}.
	
	Поставщик
	\parbox[t]{15em}{\centering{}ООО "Фирма"\\*\hrule\smallskip\tiny{}наименование},
	именуемое в дальнейшем "Поставщик", и другая фирма и другая фирма и другая фирма.
	
	$$
		2\times\Delta\tan\alpha \le 2*\Delta\tg\mu
	$$
	
	\begin{eqnarray}
		E &=& mc^2\\
		m &=& \frac{m_0}{\sqrt{1-\frac{v^2}{c^2}}}\times 2 \times 10^{23}
	\end{eqnarray}
	
	\LaTeX{} "--- это своего рода препроцессор текста для \TeX{} "---
	программы компьютерной вёрстки. \LaTeX{} является программируемым и
	расширяемым, что позволяет автоматизировать большую часть аспектов
	набора, включая нумерацию, перекрёстные ссылки, таблицы и изображения
	(их размещение и подписи к ним), общий вид страницы, библиографию и
	многое-многое другое. \LaTeX{} был первоначально написан Лэсли Лампортом
	в 1984-м году и стал наиболее популярным способом использования \TeX{}а;
	очень мало людей сегодня пишут на оригинальном \TeX{}е. Текущей
	версией является \LaTeXe.
	
	Проверим блоки.

	{\par\vspace{\partopsep}\hbox{\vrule\hspace{0.5em}\parbox{\textwidth-0.5em}{
		Иногда используется следующий способ выделения текста:
		абзац набирается с некоторым отступом от левого поля,
		а слева от него, вровень с левым полем, печатается
		вертикальная линейка.
	}}\par\vspace{\partopsep}}
	
	И ссылки: формула 2 - см. \ref{formula2}.
	
	Поставщик
	\parbox[t]{15em}{\centering{}ООО "Фирма"\\*\hrule\smallskip\tiny{}наименование},
	именуемое в дальнейшем "Поставщик", и другая фирма и другая фирма и другая фирма.
	
	$$
		2\times\Delta\tan\alpha \le 2*\Delta\tg\mu
	$$
		
	\begin{eqnarray}
		E &=& mc^2\\
		m &=& \frac{m_0}{\sqrt{1-\frac{v^2}{c^2}}}\times 2 \times 10^{23}
	\end{eqnarray}
