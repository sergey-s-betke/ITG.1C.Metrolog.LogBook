
%\setmathfont{OpenGost Type B TT}
%\setmathfont[range={\mathit}]{OpenGost Type B TT}
%\setmathfont[range={\mathup}]{OpenGost Type B TT}
%\setmathfont[range={\mathbin}]{OpenGost Type B TT}
% меняем знаки неравенств на привычные "русские"
\let\le\leqslant
\let\ge\geqslant
% определяем привычный тангенс и котангенс
\let\tan\tg
\let\arctan\arctg
\let\cot\ctg
\let\arccot\arcctg
% заменяем знак умножения (x) на точку (по ГОСТ)
\let\times\cdot

%\usepackage{forloop}

\makeatletter

\newcounter{GSI@measurementTool@cnt}
\newcounter{GSI@measurementTool@channel@cnt}[GSI@measurementTool@cnt]

\newcommand{\measurementTool}[1]{
	\addtocounter{GSI@measurementTool@cnt}{1}

	{
		\newcommand{\channel}[1]{
			\addtocounter{GSI@measurementTool@channel@cnt}{1}
			\global\@namedef{GSI@measurementTool@\number\value{GSI@measurementTool@cnt}@channel@\number\value{GSI@measurementTool@channel@cnt}}{ ##1 }
			\global\@namedef{GSI@measurementTool@\number\value{GSI@measurementTool@cnt}@channel@cnt}{\value{GSI@measurementTool@channel@cnt}}
		}

		#1
	}
}

\newcommand{\theChannel}[1]{
	\@nameuse{GSI@measurementTool@1@channel@\number#1}
}

\newcommand{\foreachChannel}[1]{
	{
		\newcounter{index}
		\forLoop{1}{\@nameuse{GSI@measurementTool@1@channel@cnt}}{index}{
			#1
		}
	}
}

\measurementTool{
	\channel{test133333}
	\channel{test2}
	\channel{test3}
}

	\newpage
	
	\object{Формуляр}.{Изделие}.{Наименование}
	\object{Формуляр}.{Разработал}[1].{ДолжностноеЛицо}.{Фамилия}
	\object{Формуляр}.{Разработал}[2].{ДолжностноеЛицо}.{Фамилия}
	
	\forEachItem{\object{Формуляр}.{Разработал}}{\ttt}{
		\ttt.{ДолжностноеЛицо}.{Инициалы}
	}
		
	\object{Формуляр}.{Разработал}.{count}
	\@nameuse{@Формуляр@Разработал@@count}
	
	\@nameuse{@Формуляр@Разработал@1@ДолжностноеЛицо@Фамилия}
	
	\@nameuse{@Формуляр@Разработал@2@ДолжностноеЛицо@Фамилия}
	
	\@nameuse{@Формуляр@Разработал@3@ДолжностноеЛицо@Фамилия}
	
	\@nameuse{@Формуляр@ЕдиницаИзмерения}
	
	%\foreachChannel{
	%	канал №\number\theindex: \theChannel{\theindex} \par
	%}
	%\multido{\iActor=1+1}{\NumberActors}{\Actors(\iActor) & \Dates(\iActor) \\ \hline}
	
	\LaTeX{} "--- это своего рода {\fontshape{it}\selectfont{}препроцессор} текста для \TeX{} "---
	программы компьютерной вёрстки. \LaTeX{} является программируемым и
	расширяемым, что позволяет автоматизировать большую часть аспектов
	набора, включая нумерацию, перекрёстные ссылки, таблицы и изображения
	(их размещение и подписи к ним), общий вид страницы, библиографию и
	многое-многое другое. \LaTeX{} был первоначально написан Лэсли Лампортом
	в 1984-м году и стал наиболее популярным способом использования \TeX{}а;
	очень мало людей сегодня пишут на оригинальном \TeX{}е. Текущей
	версией является \LaTeXe.
	
	Проверим блоки.
	
	{\par\vspace{\parsep}\vspace{\parskip}\hbox{\vrule\hspace{0.5em}\parbox{\textwidth-0.5em}{
		Иногда используется следующий способ выделения текста:
		абзац набирается с некоторым отступом от левого поля,
		а слева от него, вровень с левым полем, печатается
		вертикальная линейка.
	}}\vspace{\parsep}\par}
	
	И ссылки: формула 2 - см. \ref{formula2}.
	
	\begin{adjustwidth}{10mm}{10mm}
		Поставщик
		\parbox[t]{15em}{\centering{}ООО <<Фирма>>\\*\hrule\smallskip\tiny{}наименование},
		именуемое в дальнейшем <<Поставщик>>, и другая фирма и другая фирма и другая фирма.
	\end{adjustwidth}
	
	$$
		2\times\Delta\tan\alpha \le 2*\Delta\tg\mu
	$$
	  
	\begin{eqnarray}
		E &=& mc^2\\
		m &=& \frac{m_0}{\sqrt{1-\frac{v^2}{c^2}}}\times 2 \times 10^{23} \label{formula2}
	\end{eqnarray}
	
	\LaTeX{} "--- это своего рода препроцессор текста для \TeX{} "---
	программы компьютерной вёрстки. \LaTeX{} является программируемым и
	расширяемым, что позволяет автоматизировать большую часть аспектов
	набора, включая нумерацию, перекрёстные ссылки, таблицы и изображения
	(их размещение и подписи к ним), общий вид страницы, библиографию и
	многое-многое другое. \LaTeX{} был первоначально написан Лэсли Лампортом
	в 1984-м году и стал наиболее популярным способом использования \TeX{}а;
	очень мало людей сегодня пишут на оригинальном \TeX{}е. Текущей
	версией является \LaTeXe.
	
	Проверим блоки.
	
	{\par\vspace{\partopsep}\hbox{\vrule\hspace{0.5em}\parbox{\textwidth-0.5em}{
		Иногда используется следующий способ выделения текста:
		абзац набирается с некоторым отступом от левого поля,
		а слева от него, вровень с левым полем, печатается
		вертикальная линейка.
	}}\par\vspace{\partopsep}}
	
	И ссылки: формула 2 - см. \ref{formula2}.
	
	Поставщик
	\parbox[t]{15em}{\centering{}ООО "Фирма"\\*\hrule\smallskip\tiny{}наименование},
	именуемое в дальнейшем "Поставщик", и другая фирма и другая фирма и другая фирма.
	
	$$
		2\times\Delta\tan\alpha \le 2*\Delta\tg\mu
	$$
	
	\begin{eqnarray}
		E &=& mc^2\\
		m &=& \frac{m_0}{\sqrt{1-\frac{v^2}{c^2}}}\times 2 \times 10^{23}
	\end{eqnarray}
	
	\LaTeX{} "--- это своего рода препроцессор текста для \TeX{} "---
	программы компьютерной вёрстки. \LaTeX{} является программируемым и
	расширяемым, что позволяет автоматизировать большую часть аспектов
	набора, включая нумерацию, перекрёстные ссылки, таблицы и изображения
	(их размещение и подписи к ним), общий вид страницы, библиографию и
	многое-многое другое. \LaTeX{} был первоначально написан Лэсли Лампортом
	в 1984-м году и стал наиболее популярным способом использования \TeX{}а;
	очень мало людей сегодня пишут на оригинальном \TeX{}е. Текущей
	версией является \LaTeXe.
	
	Проверим блоки.

	{\par\vspace{\partopsep}\hbox{\vrule\hspace{0.5em}\parbox{\textwidth-0.5em}{
		Иногда используется следующий способ выделения текста:
		абзац набирается с некоторым отступом от левого поля,
		а слева от него, вровень с левым полем, печатается
		вертикальная линейка.
	}}\par\vspace{\partopsep}}
	
	И ссылки: формула 2 - см. \ref{formula2}.
	
	Поставщик
	\parbox[t]{15em}{\centering{}ООО "Фирма"\\*\hrule\smallskip\tiny{}наименование},
	именуемое в дальнейшем "Поставщик", и другая фирма и другая фирма и другая фирма.
	
	$$
		2\times\Delta\tan\alpha \le 2*\Delta\tg\mu
	$$
		
	\begin{eqnarray}
		E &=& mc^2\\
		m &=& \frac{m_0}{\sqrt{1-\frac{v^2}{c^2}}}\times 2 \times 10^{23}
	\end{eqnarray}
