\documentclass{article}

% !TeX spellcheck = ru_RU
% !TeX program = xelatex

%\usepackage[B]{gost2.304}
\usepackage[T2A,T1]{fontenc}
\usepackage{xecyr}
\usepackage{objects}
\usepackage{gost2.612}
\usepackage{trace}
\usepackage{afterpage}
\usepackage{pdflscape}

\setmainfont{GOST 2.304-81 type B}[
	Path = fonts/ ,
	Extension = .ttf ,
	% UprightFont = * ,
	BoldFont = * ,
	BoldFeatures = { FakeBold = 1.5 } ,
	ItalicFont = * Italic ,
	BoldItalicFont = * Italic ,
	BoldItalicFeatures = { FakeBold = 1.5 } ,
	% SlantedFont = * Italic ,
	% BoldSlantedFont = * Bold Italic,
	% SmallCapsFont = *
	Kerning = On
]
	

\const\Формуляр of class 
\Формуляр{
	\Организация{
		\Наименование{Новгородский филиал ФБУ~<<Тест-С.-Петербург>>}
		\Подразделение{Метрологический отдел поверки и калибровки СИ}
	}
	\Разработал{
		\ДолжностноеЛицо{
			\Должность{Инженер по метрологии}
			\Фамилия{Дель} \Инициалы{Н.Н.}
		}
		\Дата{2014/11/06}
	}
	\Разработал{
		\ДолжностноеЛицо{
			\Должность{Ведущий инженер по метрологии}
			\Фамилия{Богданова} \Инициалы{Т.М.}
		}
	}
	\Проверил{
		\ДолжностноеЛицо{
			\Должность{Ведущий инженер по метрологии}
			\Фамилия{Богданова} \Инициалы{Т.М.}
		}
		\Дата{2014/11/07}
	}
	\МетрологическийКонтроль{
		\ДолжностноеЛицо{
			\Должность{Начальник метрологического отдела поверки и калибровки средств измерений}
			\Фамилия{Ядрышникова} \Инициалы{И.С.}
		}
		\Дата{2014/11/07}
	}
	\Нормоконтроль{
		\ДолжностноеЛицо{
			\Должность{Инженер по стандартизации}
			\Фамилия{Наумова} \Инициалы{Т.И.}
		}
		\Дата{2014/11/07}
	}
	\Согласовал{
		\ДолжностноеЛицо{
			\Должность{Начальник метрологического отдела поверки и калибровки средств измерений}
			\Фамилия{Ядрышникова} \Инициалы{И.С.}
		}
		\Дата{2014/11/07}
	}
	\Утвердил{
		\ДолжностноеЛицо{
			\Должность{Директор филиала}
			\Фамилия{Бетке} \Инициалы{С.С.}
		}
		\Дата{2014/11/07}
	}
	\Изделие{
		\Наименование{Микрометр МК50-1 ГОСТ~6507-90}
		\КодОКП{123456}
		\Обозначение{НЦСМ.999999.001ФО}
		\ДатаИзготовления{11.05.1976}
		\Производитель{
			\Наименование{ЗАО~<<Какой-нибудь завод Эталон>>}
			\АдресСайта{http://www.csm.nov.ru}
		}
		\СредствоИзмерения{
			\ТипСИ{Микрометр~МК}
			\НомерГРСИ{51486-12}
			\СвидетельствоОРегистрацииТипаСИ{
				\Номер{333344}
				%\СрокДействия{01.01.2020}
			}
		}
		%
		\Характеристики{
			\Наименование{Масса изделия}
			\Значение{ \Значение{0,3} \ЕдиницаИзмерения{кг} }
		}
		\Характеристики{
			%\Наименование{Допускаемое отклонение температуры от $\numprint[\celsius]{20}$} % \todo - должно работать, необходимо исправить
			\Наименование{Допускаемое отклонение температуры от 20 C}
			\Значение{
				\Значение{\pm4}
				%\ЕдиницаИзмерения{\celsius} % \todo - должно работать, необходимо исправить
				\ЕдиницаИзмерения{C}
			}
		}
		\Характеристики{
			\Наименование{Допуск параллельности плоских измерительных поверхностей микрометра}
			\Значение{ \Значение{2,0} \ЕдиницаИзмерения{мкм} }
		}
		\Характеристики{
			\Наименование{Допуск плоскостности измерительных поверхностей микрометра}
			\Значение{ \Значение{0,6} \ЕдиницаИзмерения{мкм} }
		}
		\КаналыИзмерений{
			\Наименование{Измерение длины}
			\Обозначение{L}
			\Диапазон{
				\От{ \Значение{25} \ЕдиницаИзмерения{мкм} }
				\До{ \Значение{50} \ЕдиницаИзмерения{мкм} }
			}
			\НормируемыеМетрологическиеХарактеристики{
				\МетрологическаяХарактеристика{ПГ}
				\Наименование{Предел допускаемой основной погрешности}
				\Значение{ \Значение{1} \ЕдиницаИзмерения{мкм} }
			}
			%\НормируемыеМетрологическиеХарактеристики{
			%	\МетрологическаяХарактеристика{}
			%	\Наименование{Предел допускаемой относительной погрешности}
			%	\Значение{ \Значение{1} \ЕдиницаИзмерения{процент} }
			%}
		}
		\КаналыИзмерений{
			\Наименование{Измерение толщины}
			\Диапазон{
				\От{ \Значение{25} \ЕдиницаИзмерения{мкм} }
				\До{ \Значение{e-7} \ЕдиницаИзмерения{мм} }
			}
			\НормируемыеМетрологическиеХарактеристики{
				\МетрологическаяХарактеристика{ПГ}
				\Наименование{Предел допускаемой основной погрешности}
				\Значение{ \Значение{22222,555e-23} \ЕдиницаИзмерения{мкм} }
				\ВДиапазонеИзмерений{
					\От{ \Значение{25} \ЕдиницаИзмерения{мкм} }
					\До{ \Значение{50} \ЕдиницаИзмерения{мкм} }
				}
			}
			\НормируемыеМетрологическиеХарактеристики{
				\МетрологическаяХарактеристика{ПГО}
				\Наименование{Предел допускаемой относительной погрешности}
				\Значение{ \Значение{2,5e-3} \ЕдиницаИзмерения{мкм} }
				\ВДиапазонеИзмерений{
					\От{ \Значение{25} \ЕдиницаИзмерения{мкм} }
					\До{ \Значение{50} \ЕдиницаИзмерения{мм} }
				}
			}
		}
		\Документация{
			\РЭ{
				\Наименование{Паспорт}
				\Обозначение{АБВГ.999999.001ПС}
			}
		}
	}
	\Хранение{
		\ДатаПриёмки{01.01.2012} \ДатаСнятия{9.12.2014} \УсловияХранения{2}
		\Примечание{Хранение до ремонта}
	}
	\Хранение{
		\ДатаПриёмки{01.01.2015}
	}
	\СведенияОбУсловияхПриобретения{
		\Наименование{Договор купли-продажи}
		\Дата{01.12.2014}
		\Номер{2244/99}
		\Стороны{
			\Наименование{ЗАО~<<Какой-нибудь завод Эталон>>}
			\АдресСайта{http://www.csm.nov.ru}
		}
	}
}

\const\Тест of class 
\Действие{
	\НаименованиеДействия{jjjj}
}
	

\begin{document}

	\makeatletter

	\Формуляр\ИдентификационныеДанные\Изделие\Обозначение

	\edef\ttt{\Формуляр\Разработал[1]\ДолжностноеЛицо\Фамилия}
	\ttt

	\edef\ttt{\Формуляр\Разработал\count}
	\ttt
	
	\traceon \batchmode
	\Формуляр\Разработал\foreach{\Разработал}{
		\Разработал\ifexists\ДолжностноеЛицо {
			!!! \Разработал\ДолжностноеЛицо\Фамилия
		}{}
	}
	\traceoff
		
	-- \Формуляр\Разработал[1]\ДолжностноеЛицо\Фамилия
	-- \Формуляр\Разработал[2]\ДолжностноеЛицо\Фамилия
		
	%\afterpage{
		\begin{landscape}
		\Формуляр\Разработал\foreach{\Разработал}{
			!!! \Разработал\ДолжностноеЛицо\Фамилия
		}
		\end{landscape}
	%}
			
	\object{Формуляр}{ИдентификационныеДанные}{Изделие}{Наименование}
	-- \Формуляр\ИдентификационныеДанные\Изделие\Наименование

	\def\Разработал{\Формуляр\Разработал[1]}
	Разработал - \Разработал\ДолжностноеЛицо\Фамилия
	
	%\object{Формуляр}{Разработал}[1]{ДолжностноеЛицо}{Фамилия}
	%\Формуляр\Разработал[2]\ДолжностноеЛицо\Фамилия
	%\Формуляр\Разработал[3]\ДолжностноеЛицо\Фамилия
	%\Формуляр\Разработал[4]\ДолжностноеЛицо\Фамилия

	\Формуляр\Разработал\count
	--- \object{Формуляр}{Разработал}{count}

	\Формуляр\Разработал\foreach{\Разработал}{
		--\Разработал\ДолжностноеЛицо\Фамилия--
	}

	%\Формуляр\ОсновныеТехническиеДанные\Характеристика\foreach{\Характеристика}{%
	%	--\Характеристика\Наименование, \Характеристика\ЕдиницаИзмерения --
	%	\Характеристика\Значение
	%}

	\Формуляр\Разработал\for{\Разработал}[2][]{
		--\Разработал\ДолжностноеЛицо\Фамилия--
	}

	\Тест\НаименованиеДействия
	%\Тест\НаименованиеДейс

	\Формуляр\Организация\ifexists\Подразделение{
		\Формуляр\Организация\Подразделение
	}{}

	\def\Документ#1{%
		#1\Наименование%
		#1\ifexists\Обозначение{%
			\space#1\Обозначение%
		}{}%
	}

	\Формуляр\Документация\ifexists\РЭ{%
		в частности~--- \Документ{\Формуляр\Документация\РЭ}
	}{}

	\edef\ttt{\Формуляр\ИдентификационныеДанные\Изделие\Обозначение}
	\ttt

	\traceon \batchmode

	%\edef\ttt{\Формуляр\Разработал[1]\ДолжностноеЛицо\Фамилия}

	\traceoff

	%\edef\РазработалКоличество{\Формуляр\Разработал\count}
	%Разработал - \РазработалКоличество

	%\newcount\ttt
	%\ttt=\Формуляр\Разработал\count\relax
	%\the\ttt

\end{document}