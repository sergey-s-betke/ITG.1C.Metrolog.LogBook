\documentclass[twoside,pointsection]{gost2.105}

% формулы
\usepackage{amsmath,amsthm,amssymb}
\usepackage{mathtext}
\usepackage{unicode-math}
%\setmathfont{OpenGost Type B TT}
%\setmathfont[range={\mathit}]{OpenGost Type B TT}
%\setmathfont[range={\mathup}]{OpenGost Type B TT}
%\setmathfont[range={\mathbin}]{OpenGost Type B TT}
% меняем знаки неравенств на привычные "русские"
\let\le\leqslant
\let\ge\geqslant
% определяем привычный тангенс и котангенс
\let\tan\tg
\let\arctan\arctg
\let\cot\ctg
\let\arccot\arcctg
% заменяем знак умножения (x) на точку (по ГОСТ)
\let\times\cdot

% поддержка активных гиперссылок
\usepackage[unicode=true]{hyperref} 
\hypersetup{ %
	breaklinks=true 
	,bookmarks=true 
	,pdfauthor={Sergey S. Betke}
	,pdftitle={title}
	,colorlinks=false 
	% ,citecolor=blue
	% ,urlcolor=blue
	% ,linkcolor=magenta
	,pdfborder={0 0 0}
}
\urlstyle{same}  % don't use monospace font for urls

\usepackage{todo}

\ESKDdepartment{Метрологический отдел поверки и калибровки СИ}
\ESKDcompany{Новгородский филиал ФБУ <<Тест-С.-Петербург>>} 
\ESKDclassCode{Код изделия по ОКП}
\ESKDtitle{Формуляр}
\ESKDdocName{Микрометр МК50-2 ГОСТ 111-60}
\ESKDsignature{Обозначение документа}
\ESKDauthor{Богданова~Т.~М.}
\ESKDtitleApprovedBy{Директор филиала}{Бетке~С.~С.}
\ESKDtitleAgreedBy{Начальник МО}{Ядрышникова~И.~С.}
\ESKDtitleAgreedBy{Инженер по стандартизации}{Наумова~Т.~И.}
\ESKDtitleAgreedBy{Инженер по стандартизации}{Наумова~Т.~И.}
\ESKDtitleDesignedBy{Ведущий инженер по метрологии}{Богданова~Т.~М.}
\ESKDtitleDesignedBy{Инженер по метрологии}{Дель~Н.~Н.}
\ESKDdate{2014/10/24}

% \title{\LaTeX}
% \date{}
% \author{}

\usepackage{gost2.612}

\Формуляр{
	\Изделие{
		\Наименование{Микрометр МК50-2 ГОСТ~111-60}
		\УсловноеОбозначение{02}
	}
	\Изделие{
		\Наименование{Микрометр МК25-2 ГОСТ~111-60}
	}
}

\usepackage{forloop}

\makeatletter

\newcounter{GSI@measurementTool@cnt}
\newcounter{GSI@measurementTool@channel@cnt}[GSI@measurementTool@cnt]

\newcommand{\measurementTool}[1]{
	\addtocounter{GSI@measurementTool@cnt}{1}

	{
		\newcommand{\channel}[1]{
			\addtocounter{GSI@measurementTool@channel@cnt}{1}
			\global\@namedef{GSI@measurementTool@\number\value{GSI@measurementTool@cnt}@channel@\number\value{GSI@measurementTool@channel@cnt}}{ ##1 }
			\global\@namedef{GSI@measurementTool@\number\value{GSI@measurementTool@cnt}@channel@cnt}{\value{GSI@measurementTool@channel@cnt}}
		}

		#1
	}
}

\newcommand{\theChannel}[1]{
	\@nameuse{GSI@measurementTool@1@channel@\number#1}
}

\newcommand{\foreachChannel}[1]{
	{
		\newcounter{index}
		\forLoop{1}{\@nameuse{GSI@measurementTool@1@channel@cnt}}{index}{
			#1
		}
	}
}

\measurementTool{
	\channel{test133333}
	\channel{test2}
	\channel{test3}
}

\usepackage{forloop}

\begin{document}
	\maketitle
	\tableofcontents
	

	\section{Общие указания}
	
	\paragraph{Указания по эксплуатации изделия}
	
	\subparagraph{} Перед эксплуатацией изделия следует внимательно изучить
	эксплуатационную документацию на изделие, в частности - Руководство по эксплуатации\todo{Необходима подстановка обозначения руководства и его наимения}.
	\todo{Для оформления ссылок на сторонние документы необходимо определить свои
		макросы, которые обеспечат корректные пробелы и, возможно, другое оформление}
	
	\paragraph{Правила заполнения и ведения формуляра}

	\subparagraph{}\label{МОЛ} За ведение формуляра отвечает лицо, ответственное за эксплуатацию изделия.
	Ответственное за эксплуатацию изделия лицо указано в
	инвентарной карточке учёта объекта основных средств (форма №~ОС-6),
	хранящейся в регистрах бухгалтерского учёта.
	
	\subparagraph{} Формуляр следует вести как электронный формуляр (в соответствии с ГОСТ~2.612%
	\todo{уточнить правила указания ссылок на стандарты (год указывать или нет), и использовать
		макросы для формирования библиографии}%
	) с использованием программного средства <<1С:~Метрологическая служба>>.
	\todo{ввести макрос для наименования продукта и, возможно, ссылок на сайт в дальнейшем. Макрос оформить в отдельном стилевом файле}
	
	Записи следует дублировать в бумажной копии. В бумажную копию запись должна быть внесена
	в день внесения записи в электронный формуляр.
	
	\subparagraph{} За подготовку бумажной копии формуляра лицо, ответственное за эксплуатацию изделия (\ref{МОЛ}).
	
	\subparagraph{} В бумажной копии следует заполнять поля 22 и 23 основной надписи (<<Инв.~№ дубликата>> и <<Подп.~дата>>), при этом в поле 22 следует указывать регистрационный номер бумажной копии в деле <<Формуляры средств поверки>>.
	
	\subparagraph{} При необходимости бумажную копию формуляра следует заменять.
	При замене бумажной копии в поле 21 основной надписи (<<Взам.~инв.~№>>)
	должен быть указан регистрационный номер заменяемой бумажной копии.
	
	\subparagraph{} Бумажная копия формуляра должна постоянно находится в деле <<Формуляры средств поверки>>,
	хранящемся в помещении лаборатории, эксплуатирующей изделие.
	
	Ответственным лицом за ведение и хранение дела <<Формуляры средств поверки>> является руководитель
	лаборатории.
	
	\subparagraph{} Не допускаются записи в формуляре карандашом, смывающимися чернилами.
	Не допускаются подчистки.
	Неправильная запись должна быть аккуратно зачёркнута и рядом (в следующей строке) внесена новая запись.

	\subparagraph{} Новые записи следует заверять подписью лица, ответственного за эксплуатацию изделия.
	После подписи следует указать фамилию и инициалы ответственного лица (вместо подписи допускается
	проставлять личный штамп).

	\subparagraph{} При передаче изделия на другое предприятие итоговые записи по
	наработке следует заверять печатью предприятия, передающего изделие.


	\section{Основные сведения об изделии}
	
	\paragraph{Наименование изделия}: \@nameuse{@Формуляр@Изделие@1@Наименование}.
	\todo{Заменить на свой макрос}

	\paragraph{Дата изготовления}:
	\todo{Указать дату изготовления, форматировать макросом}
	
	\paragraph{Производитель}:
	\todo{Указать производителя}
	
	\paragraph{Заводской номер}:
	\todo{Указать заводской номер}
	
	\paragraph{Тип средства измерения}:
	\todo{Указать тип средства измерения, если он указан в реквизитах (то есть - если это средство измерения, а также указать номер сертификата и прочие аттрибуты типа СИ)}
	
	\paragraph{Регистрационный номер в Государственном реестре средств измерений}:
	\todo{Указать номер в ГРСИ, если он указан}
	\todo{Дать гиперссылку на сайт ФИФ на номере ГРСИ, либо вообще - на всём параграфе}
	\todo{Ввести сокращения СИ, ГРСИ, расшифровах их в соответствующем разделе}

	\paragraph{Номер свидетельства о регистрации типа средства измерения}:
	\todo{Указать номер свидетельства, если он указан}
	
	\paragraph{Срок действия свидетельства о регистрации типа средства измерения}:
	\todo{Указать срок действия свидетельства, если он указан}
	
	
	\section{Основные технические данные}
	
	\@nameuse{@Формуляр@Изделие@count}
	
	\@nameuse{@Формуляр@Изделие@1@Наименование}

	\@nameuse{@Формуляр@Изделие@2@Наименование}

	\@nameuse{@Формуляр@Изделие@3@Наименование}
	
	\@nameuse{@Формуляр@ЕдиницаИзмерения}
					
	\foreachChannel{
		канал №\number\theindex: \theChannel{\theindex} \par
	}
	%\multido{\iActor=1+1}{\NumberActors}{\Actors(\iActor) & \Dates(\iActor) \\ \hline}
	
	\LaTeX{} "--- это своего рода {\fontshape{it}\selectfont{}препроцессор} текста для \TeX{} "---
	программы компьютерной вёрстки. \LaTeX{} является программируемым и
	расширяемым, что позволяет автоматизировать большую часть аспектов
	набора, включая нумерацию, перекрёстные ссылки, таблицы и изображения
	(их размещение и подписи к ним), общий вид страницы, библиографию и
	многое-многое другое. \LaTeX{} был первоначально написан Лэсли Лампортом
	в 1984-м году и стал наиболее популярным способом использования \TeX{}а;
	очень мало людей сегодня пишут на оригинальном \TeX{}е. Текущей
	версией является \LaTeXe.
	
	Проверим блоки.
	
	{\par\vspace{\parsep}\vspace{\parskip}\hbox{\vrule\hspace{0.5em}\parbox{\textwidth-0.5em}{
		Иногда используется следующий способ выделения текста:
		абзац набирается с некоторым отступом от левого поля,
		а слева от него, вровень с левым полем, печатается
		вертикальная линейка.
	}}\vspace{\parsep}\par}
	
	И ссылки: формула 2 - см. \ref{formula2}.
	
	\begin{adjustwidth}{10mm}{10mm}
		Поставщик
		\parbox[t]{15em}{\centering{}ООО <<Фирма>>\\*\hrule\smallskip\tiny{}наименование},
		именуемое в дальнейшем <<Поставщик>>, и другая фирма и другая фирма и другая фирма.
	\end{adjustwidth}
	
	$$
		2\times\Delta\tan\alpha \le 2*\Delta\tg\mu
	$$
	  
	\begin{eqnarray}
		E &=& mc^2\\
		m &=& \frac{m_0}{\sqrt{1-\frac{v^2}{c^2}}}\times 2 \times 10^{23} \label{formula2}
	\end{eqnarray}
	
	\LaTeX{} "--- это своего рода препроцессор текста для \TeX{} "---
	программы компьютерной вёрстки. \LaTeX{} является программируемым и
	расширяемым, что позволяет автоматизировать большую часть аспектов
	набора, включая нумерацию, перекрёстные ссылки, таблицы и изображения
	(их размещение и подписи к ним), общий вид страницы, библиографию и
	многое-многое другое. \LaTeX{} был первоначально написан Лэсли Лампортом
	в 1984-м году и стал наиболее популярным способом использования \TeX{}а;
	очень мало людей сегодня пишут на оригинальном \TeX{}е. Текущей
	версией является \LaTeXe.
	
	Проверим блоки.
	
	{\par\vspace{\partopsep}\hbox{\vrule\hspace{0.5em}\parbox{\textwidth-0.5em}{
		Иногда используется следующий способ выделения текста:
		абзац набирается с некоторым отступом от левого поля,
		а слева от него, вровень с левым полем, печатается
		вертикальная линейка.
	}}\par\vspace{\partopsep}}
	
	И ссылки: формула 2 - см. \ref{formula2}.
	
	Поставщик
	\parbox[t]{15em}{\centering{}ООО "Фирма"\\*\hrule\smallskip\tiny{}наименование},
	именуемое в дальнейшем "Поставщик", и другая фирма и другая фирма и другая фирма.
	
	$$
		2\times\Delta\tan\alpha \le 2*\Delta\tg\mu
	$$
	
	\begin{eqnarray}
		E &=& mc^2\\
		m &=& \frac{m_0}{\sqrt{1-\frac{v^2}{c^2}}}\times 2 \times 10^{23}
	\end{eqnarray}
	
	\LaTeX{} "--- это своего рода препроцессор текста для \TeX{} "---
	программы компьютерной вёрстки. \LaTeX{} является программируемым и
	расширяемым, что позволяет автоматизировать большую часть аспектов
	набора, включая нумерацию, перекрёстные ссылки, таблицы и изображения
	(их размещение и подписи к ним), общий вид страницы, библиографию и
	многое-многое другое. \LaTeX{} был первоначально написан Лэсли Лампортом
	в 1984-м году и стал наиболее популярным способом использования \TeX{}а;
	очень мало людей сегодня пишут на оригинальном \TeX{}е. Текущей
	версией является \LaTeXe.
	
	Проверим блоки.

	{\par\vspace{\partopsep}\hbox{\vrule\hspace{0.5em}\parbox{\textwidth-0.5em}{
		Иногда используется следующий способ выделения текста:
		абзац набирается с некоторым отступом от левого поля,
		а слева от него, вровень с левым полем, печатается
		вертикальная линейка.
	}}\par\vspace{\partopsep}}
	
	И ссылки: формула 2 - см. \ref{formula2}.
	
	Поставщик
	\parbox[t]{15em}{\centering{}ООО "Фирма"\\*\hrule\smallskip\tiny{}наименование},
	именуемое в дальнейшем "Поставщик", и другая фирма и другая фирма и другая фирма.
	
	$$
		2\times\Delta\tan\alpha \le 2*\Delta\tg\mu
	$$
		
	\begin{eqnarray}
		E &=& mc^2\\
		m &=& \frac{m_0}{\sqrt{1-\frac{v^2}{c^2}}}\times 2 \times 10^{23}
	\end{eqnarray}
	
	\todos

\end{document}