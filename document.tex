\documentclass[twoside]{gost2.610.logbook}

\usepackage{1c-metrolog}
\usepackage{todo}

\ПравилаЗаполненияИВеденияФормуляра{
	\paragraph{}\label{МОЛ} За ведение формуляра отвечает лицо, ответственное за эксплуатацию изделия.
	Ответственное за эксплуатацию изделия лицо указано в
	инвентарной карточке учёта объекта основных средств (форма №~ОС-6),
	хранящейся в регистрах бухгалтерского учёта.
	
	\paragraph{} Формуляр следует вести как электронный формуляр (в соответствии с ГОСТ~2.612%
	\todo{уточнить правила указания ссылок на стандарты (год указывать или нет), и использовать
		макросы для формирования библиографии}%
	) с использованием программного средства \ais.
	\done\todo{ввести макрос для наименования продукта и, возможно, ссылок на сайт в дальнейшем. Макрос оформить в отдельном стилевом файле}
	
	Записи следует дублировать в бумажной копии. В бумажную копию запись должна быть внесена
	в день внесения записи в электронный формуляр.
	
	\paragraph{} За подготовку бумажной копии формуляра лицо, ответственное за эксплуатацию изделия (\ref{МОЛ}).
	
	\paragraph{} В бумажной копии следует заполнять поля 22 и 23 основной надписи (<<Инв.~№ дубликата>> и <<Подп.~дата>>), при этом в поле 22 следует указывать регистрационный номер бумажной копии в деле <<Формуляры средств поверки>>.
	
	\paragraph{} При необходимости бумажную копию формуляра следует заменять.
	При замене бумажной копии в поле 21 основной надписи (<<Взам.~инв.~№>>)
	должен быть указан регистрационный номер заменяемой бумажной копии.
	
	\paragraph{} Бумажная копия формуляра должна постоянно находится в деле <<Формуляры средств поверки>>,
	хранящемся в помещении лаборатории, эксплуатирующей изделие.
	
	Ответственным лицом за ведение и хранение дела <<Формуляры средств поверки>> является руководитель
	лаборатории.
	
	\paragraph{} Не допускаются записи в формуляре карандашом, смывающимися чернилами.
	Не допускаются подчистки.
	Неправильная запись должна быть аккуратно зачёркнута и рядом (в следующей строке) внесена новая запись.
	
	\paragraph{} Новые записи следует заверять подписью лица, ответственного за эксплуатацию изделия.
	После подписи следует указать фамилию и инициалы ответственного лица (вместо подписи допускается
	проставлять личный штамп).
	
	\paragraph{} При передаче изделия на другое предприятие итоговые записи по
	наработке следует заверять печатью предприятия, передающего изделие.
}

\Формуляр{
	\Организация{
		\Наименование{Новгородский филиал ФБУ~<<Тест-С.-Петербург>>}
		\Подразделение{Метрологический отдел поверки и калибровки СИ}
	}
	\Изделие{
		\Наименование{Микрометр МК50-2 ГОСТ~111-60}
		\КодОКП{123456}
	}
}


\begin{document}
	
	
%\setmathfont{OpenGost Type B TT}
%\setmathfont[range={\mathit}]{OpenGost Type B TT}
%\setmathfont[range={\mathup}]{OpenGost Type B TT}
%\setmathfont[range={\mathbin}]{OpenGost Type B TT}
% меняем знаки неравенств на привычные "русские"
\let\le\leqslant
\let\ge\geqslant
% определяем привычный тангенс и котангенс
\let\tan\tg
\let\arctan\arctg
\let\cot\ctg
\let\arccot\arcctg
% заменяем знак умножения (x) на точку (по ГОСТ)
\let\times\cdot

\makeatletter

	\newpage
	
	\ПГ(\frac{L}{2^{64}+\int\limits_{0}^{2\pi}f\left(x\right)dx}), \ПГ(L), \СКО(L), \ПГО(L).
	
	$\numprint{+-3,1415927e-3.1}$~--- пробуем numprint. \numprint{-22978}, \numprint{-3.72e26},
	$\numprint[\celsius]{25}$, $\numprint[\degree]{25}$.
	
	Тестируем siunitx: \SIrange{1}{10}{\degreeCelsius}. % [mode=text, detect-all]

	\object{Формуляр}{ИдентификационныеДанные}{Изделие}{Наименование}
	\Формуляр\ИдентификационныеДанные\Изделие\Наименование

	\def\Разработал{\Формуляр\Разработал[1]}
	Разработал - \Разработал\ДолжностноеЛицо\Фамилия

	\object{Формуляр}{Разработал}[1]{ДолжностноеЛицо}{Фамилия}
	\Формуляр\Разработал[2]\ДолжностноеЛицо\Фамилия
	%\Формуляр\Разработал[3]\ДолжностноеЛицо\Фамилия
	%\Формуляр\Разработал[4]\ДолжностноеЛицо\Фамилия
		
	\Формуляр\Разработал\count
	--- \object{Формуляр}{Разработал}\count
	--- \@nameuse{@Формуляр@Разработал@count}

	%\Формуляр\Разработал\foreach{\Разработал}{
	%	--\Разработал\ДолжностноеЛицо\Фамилия--
	%}

	\Формуляр\ОсновныеТехническиеДанные\Характеристика\foreach{\Характеристика}{%
		--\Характеристика\Наименование, \Характеристика\Значение\ЕдиницаИзмерения --
		\Характеристика\Значение\Значение
	}
	
	\Формуляр\Разработал\for{\Разработал}[2][]{
		--\Разработал\ДолжностноеЛицо\Фамилия--
	}
		
	\LaTeX{} "--- это своего рода {\fontshape{it}\selectfont{}препроцессор} текста для \TeX{} "---
	программы компьютерной вёрстки. \LaTeX{} является программируемым и
	расширяемым, что позволяет автоматизировать большую часть аспектов
	набора, включая нумерацию, перекрёстные ссылки, таблицы и изображения
	(их размещение и подписи к ним), общий вид страницы, библиографию и
	многое-многое другое. \LaTeX{} был первоначально написан Лэсли Лампортом
	в 1984-м году и стал наиболее популярным способом использования \TeX{}а;
	очень мало людей сегодня пишут на оригинальном \TeX{}е. Текущей
	версией является \LaTeXe.
	
	Проверим блоки.
	
	{\par\vspace{\parsep}\vspace{\parskip}\hbox{\vrule\hspace{0.5em}\parbox{\textwidth-0.5em}{
		Иногда используется следующий способ выделения текста:
		абзац набирается с некоторым отступом от левого поля,
		а слева от него, вровень с левым полем, печатается
		вертикальная линейка.
	}}\vspace{\parsep}\par}
	
	И ссылки: формула 2 - см. \ref{formula2}.
	
	\begin{adjustwidth}{10mm}{10mm}
		Поставщик
		\parbox[t]{15em}{\centering{}ООО <<Фирма>>\\*\hrule\smallskip\tiny{}наименование},
		именуемое в дальнейшем <<Поставщик>>, и другая фирма и другая фирма и другая фирма.
	\end{adjustwidth}
	
	$$
		2\times\Delta\tan\alpha \le 2*\Delta\tg\mu
	$$
	  
	\begin{eqnarray}
		E &=& mc^2\\
		m &=& \frac{m_0}{\sqrt{1-\frac{v^2}{c^2}}}\times 2 \times 10^{23} \label{formula2}
	\end{eqnarray}
	
	\LaTeX{} "--- это своего рода препроцессор текста для \TeX{} "---
	программы компьютерной вёрстки. \LaTeX{} является программируемым и
	расширяемым, что позволяет автоматизировать большую часть аспектов
	набора, включая нумерацию, перекрёстные ссылки, таблицы и изображения
	(их размещение и подписи к ним), общий вид страницы, библиографию и
	многое-многое другое. \LaTeX{} был первоначально написан Лэсли Лампортом
	в 1984-м году и стал наиболее популярным способом использования \TeX{}а;
	очень мало людей сегодня пишут на оригинальном \TeX{}е. Текущей
	версией является \LaTeXe.
	
	Проверим блоки.
	
	{\par\vspace{\partopsep}\hbox{\vrule\hspace{0.5em}\parbox{\textwidth-0.5em}{
		Иногда используется следующий способ выделения текста:
		абзац набирается с некоторым отступом от левого поля,
		а слева от него, вровень с левым полем, печатается
		вертикальная линейка.
	}}\par\vspace{\partopsep}}
	
	И ссылки: формула 2 - см. \ref{formula2}.
	
	Поставщик
	\parbox[t]{15em}{\centering{}ООО "Фирма"\\*\hrule\smallskip\tiny{}наименование},
	именуемое в дальнейшем "Поставщик", и другая фирма и другая фирма и другая фирма.
	
	$$
		2\times\Delta\tan\alpha \le 2*\Delta\tg\mu
	$$
	
	\begin{eqnarray}
		E &=& mc^2\\
		m &=& \frac{m_0}{\sqrt{1-\frac{v^2}{c^2}}}\times 2 \times 10^{23}
	\end{eqnarray}
	
	\LaTeX{} "--- это своего рода препроцессор текста для \TeX{} "---
	программы компьютерной вёрстки. \LaTeX{} является программируемым и
	расширяемым, что позволяет автоматизировать большую часть аспектов
	набора, включая нумерацию, перекрёстные ссылки, таблицы и изображения
	(их размещение и подписи к ним), общий вид страницы, библиографию и
	многое-многое другое. \LaTeX{} был первоначально написан Лэсли Лампортом
	в 1984-м году и стал наиболее популярным способом использования \TeX{}а;
	очень мало людей сегодня пишут на оригинальном \TeX{}е. Текущей
	версией является \LaTeXe.
	
	Проверим блоки.

	{\par\vspace{\partopsep}\hbox{\vrule\hspace{0.5em}\parbox{\textwidth-0.5em}{
		Иногда используется следующий способ выделения текста:
		абзац набирается с некоторым отступом от левого поля,
		а слева от него, вровень с левым полем, печатается
		вертикальная линейка.
	}}\par\vspace{\partopsep}}
	
	И ссылки: формула 2 - см. \ref{formula2}.
	
	Поставщик
	\parbox[t]{15em}{\centering{}ООО "Фирма"\\*\hrule\smallskip\tiny{}наименование},
	именуемое в дальнейшем "Поставщик", и другая фирма и другая фирма и другая фирма.
	
	$$
		2\times\Delta\tan\alpha \le 2*\Delta\tg\mu
	$$
		
	\begin{eqnarray}
		E &=& mc^2\\
		m &=& \frac{m_0}{\sqrt{1-\frac{v^2}{c^2}}}\times 2 \times 10^{23}
	\end{eqnarray}

	
\end{document}