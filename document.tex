\documentclass[twoside]{gost2.610.logbook}

% !TeX spellcheck = ru_RU
% !TeX program = xelatex

\usepackage{1c-metrolog}
\usepackage{todo}

\ПравилаЗаполненияИВеденияФормуляра{
	\paragraph{}\label{МОЛ} За ведение формуляра отвечает лицо, ответственное за эксплуатацию изделия.
	Ответственное за эксплуатацию изделия лицо указано в
	инвентарной карточке учёта объекта основных средств (форма №~ОС-6),
	хранящейся в регистрах бухгалтерского учёта.
	
	\paragraph{} Формуляр следует вести как электронный формуляр (в соответствии с ГОСТ~2.612%
	\todo{уточнить правила указания ссылок на стандарты (год указывать или нет), и использовать
		макросы для формирования библиографии}%
	) с использованием программного средства \ais.
	\done\todo{ввести макрос для наименования продукта и, возможно, ссылок на сайт в дальнейшем. Макрос оформить в отдельном стилевом файле}
	
	Записи следует дублировать в бумажной копии. В бумажную копию запись должна быть внесена
	в день внесения записи в электронный формуляр.
	
	\paragraph{} За подготовку бумажной копии формуляра лицо, ответственное за эксплуатацию изделия (\ref{МОЛ}).
	
	\paragraph{} В бумажной копии следует заполнять поля 22 и 23 основной надписи (<<Инв.~№ дубликата>> и <<Подп.~дата>>), при этом в поле 22 следует указывать регистрационный номер бумажной копии в деле <<Формуляры средств поверки>>.
	
	\paragraph{} При необходимости бумажную копию формуляра следует заменять.
	При замене бумажной копии в поле 21 основной надписи (<<Взам.~инв.~№>>)
	должен быть указан регистрационный номер заменяемой бумажной копии.
	
	\paragraph{} Бумажная копия формуляра должна постоянно находится в деле <<Формуляры средств поверки>>,
	хранящемся в помещении лаборатории, эксплуатирующей изделие.
	
	Ответственным лицом за ведение и хранение дела <<Формуляры средств поверки>> является руководитель
	лаборатории.
	
	\paragraph{} Не допускаются записи в формуляре карандашом, смывающимися чернилами.
	Не допускаются подчистки.
	Неправильная запись должна быть аккуратно зачёркнута и рядом (в следующей строке) внесена новая запись.
	
	\paragraph{} Новые записи следует заверять подписью лица, ответственного за эксплуатацию изделия.
	После подписи следует указать фамилию и инициалы ответственного лица (вместо подписи допускается
	проставлять личный штамп).
	
	\paragraph{} При передаче изделия на другое предприятие итоговые записи по
	наработке следует заверять печатью предприятия, передающего изделие.
}

\const\Формуляр of class 
\Формуляр{
	\Организация{
		\Наименование{Новгородский филиал ФБУ~<<Тест-С.-Петербург>>}
		\Подразделение{Метрологический отдел поверки и калибровки СИ}
	}
	\Разработал{
		\ДолжностноеЛицо{
			\Должность{Инженер по метрологии}
			\Фамилия{Дель} \Инициалы{Н.Н.}
		}
		\Дата{2014/11/06}
	}
	\Разработал{
		\ДолжностноеЛицо{
			\Должность{Ведущий инженер по метрологии}
			\Фамилия{Богданова} \Инициалы{Т.М.}
		}
	}
	\Проверил{
		\ДолжностноеЛицо{
			\Должность{Ведущий инженер по метрологии}
			\Фамилия{Богданова} \Инициалы{Т.М.}
		}
		\Дата{2014/11/07}
	}
	\МетрологическийКонтроль{
		\ДолжностноеЛицо{
			\Должность{Начальник метрологического отдела поверки и калибровки средств измерений}
			\Фамилия{Ядрышникова} \Инициалы{И.С.}
		}
		\Дата{2014/11/07}
	}
	\Нормоконтроль{
		\ДолжностноеЛицо{
			\Должность{Инженер по стандартизации}
			\Фамилия{Наумова} \Инициалы{Т.И.}
		}
		\Дата{2014/11/07}
	}
	\Согласовал{
		\ДолжностноеЛицо{
			\Должность{Начальник метрологического отдела поверки и калибровки средств измерений}
			\Фамилия{Ядрышникова} \Инициалы{И.С.}
		}
		\Дата{2014/11/07}
	}
	\Утвердил{
		\ДолжностноеЛицо{
			\Должность{Директор филиала}
			\Фамилия{Бетке} \Инициалы{С.С.}
		}
		\Дата{2014/11/07}
	}
	\Изделие{
		\Наименование{Микрометр МК50-1 ГОСТ~6507-90}
		\КодОКП{123456}
		\Обозначение{НЦСМ.999999.001ФО}
		\ДатаИзготовления{11.05.1976}
		\Производитель{
			\Наименование{ЗАО~<<Какой-нибудь завод Эталон>>}
			\АдресСайта{http://www.csm.nov.ru}
		}
		\СредствоИзмерения{
			\ТипСИ{Микрометр~МК}
			\НомерГРСИ{51486-12}
			\СвидетельствоОРегистрацииТипаСИ{
				\Номер{333344}
				%\СрокДействия{01.01.2020}
			}
		}
		%
		\Характеристики{
			\Наименование{Масса изделия}
			\Значение{ \Значение{0,3} \ЕдиницаИзмерения{кг} }
		}
		\Характеристики{
			%\Наименование{Допускаемое отклонение температуры от $\numprint[\celsius]{20}$} % \todo - должно работать, необходимо исправить
			\Наименование{Допускаемое отклонение температуры от 20 C}
			\Значение{
				\Значение{\pm4}
				%\ЕдиницаИзмерения{\celsius} % \todo - должно работать, необходимо исправить
				\ЕдиницаИзмерения{C}
			}
		}
		\Характеристики{
			\Наименование{Допуск параллельности плоских измерительных поверхностей микрометра}
			\Значение{ \Значение{2,0} \ЕдиницаИзмерения{мкм} }
		}
		\Характеристики{
			\Наименование{Допуск плоскостности измерительных поверхностей микрометра}
			\Значение{ \Значение{0,6} \ЕдиницаИзмерения{мкм} }
		}
		\КаналыИзмерений{
			\Наименование{Измерение длины}
			\Обозначение{L}
			\Диапазон{
				\От{ \Значение{25} \ЕдиницаИзмерения{мкм} }
				\До{ \Значение{50} \ЕдиницаИзмерения{мкм} }
			}
			\НормируемыеМетрологическиеХарактеристики{
				\МетрологическаяХарактеристика{ПГ}
				\Наименование{Предел допускаемой основной погрешности}
				\Значение{ \Значение{1} \ЕдиницаИзмерения{мкм} }
			}
			%\НормируемыеМетрологическиеХарактеристики{
			%	\МетрологическаяХарактеристика{}
			%	\Наименование{Предел допускаемой относительной погрешности}
			%	\Значение{ \Значение{1} \ЕдиницаИзмерения{процент} }
			%}
		}
		\КаналыИзмерений{
			\Наименование{Измерение толщины}
			\Диапазон{
				\От{ \Значение{25} \ЕдиницаИзмерения{мкм} }
				\До{ \Значение{e-7} \ЕдиницаИзмерения{мм} }
			}
			\НормируемыеМетрологическиеХарактеристики{
				\МетрологическаяХарактеристика{ПГ}
				\Наименование{Предел допускаемой основной погрешности}
				\Значение{ \Значение{22222,555e-23} \ЕдиницаИзмерения{мкм} }
				\ВДиапазонеИзмерений{
					\От{ \Значение{25} \ЕдиницаИзмерения{мкм} }
					\До{ \Значение{50} \ЕдиницаИзмерения{мкм} }
				}
			}
			\НормируемыеМетрологическиеХарактеристики{
				\МетрологическаяХарактеристика{ПГО}
				\Наименование{Предел допускаемой относительной погрешности}
				\Значение{ \Значение{2,5e-3} \ЕдиницаИзмерения{мкм} }
				\ВДиапазонеИзмерений{
					\От{ \Значение{25} \ЕдиницаИзмерения{мкм} }
					\До{ \Значение{50} \ЕдиницаИзмерения{мм} }
				}
			}
		}
		\Документация{
			\РЭ{
				\Наименование{Паспорт}
				\Обозначение{АБВГ.999999.001ПС}
			}
		}
	}
	\Хранение{
		\ДатаПриёмки{01.01.2012} \ДатаСнятия{9.12.2014} \УсловияХранения{2}
		\Примечание{Хранение до ремонта}
	}
	\Хранение{
		\ДатаПриёмки{01.01.2015}
	}
	\СведенияОбУсловияхПриобретения{
		\Наименование{Договор купли-продажи}
		\Дата{01.12.2014}
		\Номер{2244/99}
		\Стороны{
			\Наименование{ЗАО~<<Какой-нибудь завод Эталон>>}
			\АдресСайта{http://www.csm.nov.ru}
		}
	}
}

\begin{document}
	
	
%\setmathfont{OpenGost Type B TT}
%\setmathfont[range={\mathit}]{OpenGost Type B TT}
%\setmathfont[range={\mathup}]{OpenGost Type B TT}
%\setmathfont[range={\mathbin}]{OpenGost Type B TT}
% меняем знаки неравенств на привычные "русские"
\let\le\leqslant
\let\ge\geqslant
% определяем привычный тангенс и котангенс
\let\tan\tg
\let\arctan\arctg
\let\cot\ctg
\let\arccot\arcctg
% заменяем знак умножения (x) на точку (по ГОСТ)
\let\times\cdot

%\usepackage{forloop}

\makeatletter

\newcounter{GSI@measurementTool@cnt}
\newcounter{GSI@measurementTool@channel@cnt}[GSI@measurementTool@cnt]

\newcommand{\measurementTool}[1]{
	\addtocounter{GSI@measurementTool@cnt}{1}

	{
		\newcommand{\channel}[1]{
			\addtocounter{GSI@measurementTool@channel@cnt}{1}
			\global\@namedef{GSI@measurementTool@\number\value{GSI@measurementTool@cnt}@channel@\number\value{GSI@measurementTool@channel@cnt}}{ ##1 }
			\global\@namedef{GSI@measurementTool@\number\value{GSI@measurementTool@cnt}@channel@cnt}{\value{GSI@measurementTool@channel@cnt}}
		}

		#1
	}
}

\newcommand{\theChannel}[1]{
	\@nameuse{GSI@measurementTool@1@channel@\number#1}
}

\newcommand{\foreachChannel}[1]{
	{
		\newcounter{index}
		\forLoop{1}{\@nameuse{GSI@measurementTool@1@channel@cnt}}{index}{
			#1
		}
	}
}

\measurementTool{
	\channel{test133333}
	\channel{test2}
	\channel{test3}
}

	\newpage
	
	\object{Формуляр}.{Изделие}.{Наименование}
	\object{Формуляр}.{Разработал}[1].{ДолжностноеЛицо}.{Фамилия}
	\object{Формуляр}.{Разработал}[2].{ДолжностноеЛицо}.{Фамилия}
		
	\@nameuse{@Формуляр@Разработал@count}
	
	\@nameuse{@Формуляр@Разработал@1@ДолжностноеЛицо@Фамилия}
	
	\@nameuse{@Формуляр@Разработал@2@ДолжностноеЛицо@Фамилия}
	
	\@nameuse{@Формуляр@Разработал@3@ДолжностноеЛицо@Фамилия}
	
	\@nameuse{@Формуляр@ЕдиницаИзмерения}
	
	%\foreachChannel{
	%	канал №\number\theindex: \theChannel{\theindex} \par
	%}
	%\multido{\iActor=1+1}{\NumberActors}{\Actors(\iActor) & \Dates(\iActor) \\ \hline}
	
	\LaTeX{} "--- это своего рода {\fontshape{it}\selectfont{}препроцессор} текста для \TeX{} "---
	программы компьютерной вёрстки. \LaTeX{} является программируемым и
	расширяемым, что позволяет автоматизировать большую часть аспектов
	набора, включая нумерацию, перекрёстные ссылки, таблицы и изображения
	(их размещение и подписи к ним), общий вид страницы, библиографию и
	многое-многое другое. \LaTeX{} был первоначально написан Лэсли Лампортом
	в 1984-м году и стал наиболее популярным способом использования \TeX{}а;
	очень мало людей сегодня пишут на оригинальном \TeX{}е. Текущей
	версией является \LaTeXe.
	
	Проверим блоки.
	
	{\par\vspace{\parsep}\vspace{\parskip}\hbox{\vrule\hspace{0.5em}\parbox{\textwidth-0.5em}{
		Иногда используется следующий способ выделения текста:
		абзац набирается с некоторым отступом от левого поля,
		а слева от него, вровень с левым полем, печатается
		вертикальная линейка.
	}}\vspace{\parsep}\par}
	
	И ссылки: формула 2 - см. \ref{formula2}.
	
	\begin{adjustwidth}{10mm}{10mm}
		Поставщик
		\parbox[t]{15em}{\centering{}ООО <<Фирма>>\\*\hrule\smallskip\tiny{}наименование},
		именуемое в дальнейшем <<Поставщик>>, и другая фирма и другая фирма и другая фирма.
	\end{adjustwidth}
	
	$$
		2\times\Delta\tan\alpha \le 2*\Delta\tg\mu
	$$
	  
	\begin{eqnarray}
		E &=& mc^2\\
		m &=& \frac{m_0}{\sqrt{1-\frac{v^2}{c^2}}}\times 2 \times 10^{23} \label{formula2}
	\end{eqnarray}
	
	\LaTeX{} "--- это своего рода препроцессор текста для \TeX{} "---
	программы компьютерной вёрстки. \LaTeX{} является программируемым и
	расширяемым, что позволяет автоматизировать большую часть аспектов
	набора, включая нумерацию, перекрёстные ссылки, таблицы и изображения
	(их размещение и подписи к ним), общий вид страницы, библиографию и
	многое-многое другое. \LaTeX{} был первоначально написан Лэсли Лампортом
	в 1984-м году и стал наиболее популярным способом использования \TeX{}а;
	очень мало людей сегодня пишут на оригинальном \TeX{}е. Текущей
	версией является \LaTeXe.
	
	Проверим блоки.
	
	{\par\vspace{\partopsep}\hbox{\vrule\hspace{0.5em}\parbox{\textwidth-0.5em}{
		Иногда используется следующий способ выделения текста:
		абзац набирается с некоторым отступом от левого поля,
		а слева от него, вровень с левым полем, печатается
		вертикальная линейка.
	}}\par\vspace{\partopsep}}
	
	И ссылки: формула 2 - см. \ref{formula2}.
	
	Поставщик
	\parbox[t]{15em}{\centering{}ООО "Фирма"\\*\hrule\smallskip\tiny{}наименование},
	именуемое в дальнейшем "Поставщик", и другая фирма и другая фирма и другая фирма.
	
	$$
		2\times\Delta\tan\alpha \le 2*\Delta\tg\mu
	$$
	
	\begin{eqnarray}
		E &=& mc^2\\
		m &=& \frac{m_0}{\sqrt{1-\frac{v^2}{c^2}}}\times 2 \times 10^{23}
	\end{eqnarray}
	
	\LaTeX{} "--- это своего рода препроцессор текста для \TeX{} "---
	программы компьютерной вёрстки. \LaTeX{} является программируемым и
	расширяемым, что позволяет автоматизировать большую часть аспектов
	набора, включая нумерацию, перекрёстные ссылки, таблицы и изображения
	(их размещение и подписи к ним), общий вид страницы, библиографию и
	многое-многое другое. \LaTeX{} был первоначально написан Лэсли Лампортом
	в 1984-м году и стал наиболее популярным способом использования \TeX{}а;
	очень мало людей сегодня пишут на оригинальном \TeX{}е. Текущей
	версией является \LaTeXe.
	
	Проверим блоки.

	{\par\vspace{\partopsep}\hbox{\vrule\hspace{0.5em}\parbox{\textwidth-0.5em}{
		Иногда используется следующий способ выделения текста:
		абзац набирается с некоторым отступом от левого поля,
		а слева от него, вровень с левым полем, печатается
		вертикальная линейка.
	}}\par\vspace{\partopsep}}
	
	И ссылки: формула 2 - см. \ref{formula2}.
	
	Поставщик
	\parbox[t]{15em}{\centering{}ООО "Фирма"\\*\hrule\smallskip\tiny{}наименование},
	именуемое в дальнейшем "Поставщик", и другая фирма и другая фирма и другая фирма.
	
	$$
		2\times\Delta\tan\alpha \le 2*\Delta\tg\mu
	$$
		
	\begin{eqnarray}
		E &=& mc^2\\
		m &=& \frac{m_0}{\sqrt{1-\frac{v^2}{c^2}}}\times 2 \times 10^{23}
	\end{eqnarray}

	
\end{document}