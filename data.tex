\const\Формуляр of class 
\Формуляр{
	\Организация{
		\Наименование{Новгородский филиал ФБУ~<<Тест-С.-Петербург>>}
		\Подразделение{Метрологический отдел поверки и калибровки СИ}
	}
	\Разработал{
		\ДолжностноеЛицо{
			\Должность{Инженер по метрологии}
			\Фамилия{Дель} \Инициалы{Н.Н.}
		}
		\Дата{2014/11/06}
	}
	\Разработал{
		\ДолжностноеЛицо{
			\Должность{Ведущий инженер по метрологии}
			\Фамилия{Богданова} \Инициалы{Т.М.}
		}
	}
	\Проверил{
		\ДолжностноеЛицо{
			\Должность{Ведущий инженер по метрологии}
			\Фамилия{Богданова} \Инициалы{Т.М.}
		}
		\Дата{2014/11/07}
	}
	\МетрологическийКонтроль{
		\ДолжностноеЛицо{
			\Должность{Начальник метрологического отдела поверки и калибровки средств измерений}
			\Фамилия{Ядрышникова} \Инициалы{И.С.}
		}
		\Дата{2014/11/07}
	}
	\Нормоконтроль{
		\ДолжностноеЛицо{
			\Должность{Инженер по стандартизации}
			\Фамилия{Наумова} \Инициалы{Т.И.}
		}
		\Дата{2014/11/07}
	}
	\Согласовал{
		\ДолжностноеЛицо{
			\Должность{Начальник метрологического отдела поверки и калибровки средств измерений}
			\Фамилия{Ядрышникова} \Инициалы{И.С.}
		}
		\Дата{2014/11/07}
	}
	\Утвердил{
		\ДолжностноеЛицо{
			\Должность{Директор филиала}
			\Фамилия{Бетке} \Инициалы{С.С.}
		}
		\Дата{2014/11/07}
	}
	\ИдентификационныеДанные{
		\Изделие{
			\Наименование{Микрометр МК50-1 ГОСТ~6507-90}
			\КодОКП{123456}
			\Обозначение{НЦСМ.999999.001ФО}
		}
		\ДатаИзготовления{11.05.1976}
		\Производитель{
			\Наименование{ЗАО~<<Какой-нибудь завод Эталон>>}
			\АдресСайта{http://www.csm.nov.ru}
		}
		\СредствоИзмерения{
			\ТипСИ{Микрометр~МК}
			\НомерГРСИ{51486-12}
			\СвидетельствоОРегистрацииТипаСИ{
				\Номер{333344}
				%\СрокДействия{01.01.2020}
			}
		}
	}
	\ОсновныеТехническиеДанные{
		\Характеристика{
			\Наименование{Масса изделия}
			\Значение{ \Значение{0,3} \ЕдиницаИзмерения{кг} }
		}
		\Характеристика{
			%\Наименование{Допускаемое отклонение температуры от $\numprint[\celsius]{20}$} % \todo - должно работать, необходимо исправить
			\Наименование{Допускаемое отклонение температуры от 20 C}
			\Значение{
				\Значение{\pm4}
				%\ЕдиницаИзмерения{\celsius} % \todo - должно работать, необходимо исправить
				\ЕдиницаИзмерения{C}
			}
		}
		\Характеристика{
			\Наименование{Допуск параллельности плоских измерительных поверхностей микрометра}
			\Значение{ \Значение{2,0} \ЕдиницаИзмерения{мкм} }
		}
		\Характеристика{
			\Наименование{Допуск плоскостности измерительных поверхностей микрометра}
			\Значение{ \Значение{0,6} \ЕдиницаИзмерения{мкм} }
		}
		\КаналИзмерений{
			\Наименование{Измерение длины}
			\Диапазон{
				\От{ \Значение{25} \ЕдиницаИзмерения{мкм} }
				\До{ \Значение{50} \ЕдиницаИзмерения{мкм} }
			}
			\НормируемаяМетрологическаяХарактеристика{
				\МетрологическаяХарактеристика{}
				\Наименование{Предел допускаемой основной погрешности}
				\Значение{ \Значение{1} \ЕдиницаИзмерения{мкм} }
				%\ВДиапазонеИзмерений{
				%	\От{ \Значение{25} \ЕдиницаИзмерения{мкм} }
				%	\До{ \Значение{50} \ЕдиницаИзмерения{мкм} }
				%}
			}
			\НормируемаяМетрологическаяХарактеристика{
				\МетрологическаяХарактеристика{}
				\Наименование{Предел допускаемой относительной погрешности}
				\Значение{ \Значение{1} \ЕдиницаИзмерения{процент} }
			}
		}
		\КаналИзмерений{
			\Наименование{Измерение толщины}
			\Диапазон{
				\От{ \Значение{25} \ЕдиницаИзмерения{мкм} }
				\До{ \Значение{50} \ЕдиницаИзмерения{мм} }
			}
			\НормируемаяМетрологическаяХарактеристика{
				\МетрологическаяХарактеристика{}
				\Наименование{Предел допускаемой основной погрешности}
				\Значение{ \Значение{1} \ЕдиницаИзмерения{мкм} }
				%\ВДиапазонеИзмерений{
				%	\От{ \Значение{25} \ЕдиницаИзмерения{мкм} }
				%	\До{ \Значение{50} \ЕдиницаИзмерения{мкм} }
				%}
			}
		}
	}
	\Документация{
		\РЭ{
			\Наименование{Паспорт}
			\Обозначение{АБВГ.999999.001ПС}
		}
	}
}